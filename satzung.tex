\documentclass[10pt,conference,a4paper,nofonttune]{IEEEtran}

\usepackage[utf8]{inputenc}
\usepackage{ngerman}
\usepackage{hyperref}
\usepackage[pdftex]{graphicx}
\usepackage{xcolor}

\renewcommand{\thesection}{\S \arabic{section}}

\renewcommand{\abstractname}{\vspace{-\baselineskip}}

\begin{document}

\title{Satzung Hackspace Marburg e.V.}


\twocolumn[
  \begin{@twocolumnfalse}
    \maketitle

    \begin{abstract}
      Die Informationsgesellschaft unserer Tage ist ohne Computer nicht mehr
      denkbar. Die Einsatzmöglichkeiten der automatisierten Datenverarbeitung
      und Datenübermittlung bergen Chancen, aber auch Gefahren für den Einzelnen
      und für die Gesellschaft. Informations- und Kommunikationstechnologien
      verändern das Verhältnis Mensch-Maschine und der Menschen untereinander.
      Die Entwicklung zur Informationsgesellschaft erfordert ein neues
      Menschenrecht auf weltweite, ungehinderte Kommunikation.

      Wir sind eine Gemeinschaft von Menschen, unabhängig von Alter, Geschlecht
      und Herkunft sowie gesellschaftlicher Stellung, die sich
      grenzüberschreitend für Informationsfreiheit einsetzt und mit den
      Auswirkungen von Technologien auf die Gesellschaft sowie das einzelne
      Lebewesen beschäftigt und das Wissen um diese Entwicklung fördert.
    \end{abstract}
  \end{@twocolumnfalse}
]


\section{Name, Sitz}
\begin{enumerate}
  \item Der Verein führt den Namen \textit{Hackspace Marburg e.V.}

  \item Er soll in das Vereinsregister eingetragen werden und führt danach den
    Zusatz e.V.

  \item Der Sitz des Vereins ist in Marburg.
\end{enumerate}


\section{Zweck}
\begin{enumerate}
  \item Der Verein ist parteipolitisch und weltanschaulich neutral.

  \item Der Verein setzt sich zum Zweck:
    \begin{itemize}
      \item die Förderung der Erziehung und Volksbildung, insbesondere der
        Informatik- und Medienkompetenz der breiten Öffentlichkeit, sowie
        Aufklärung über und kritische Betrachtung von Risiken und Möglichkeiten
        neuer Technologien.

      \item Kunst und Kultur in Hinblick auf den schöpferischen Umgang mit
        Technologie zu fördern.
    \end{itemize}

  \item Der Vereinszweck soll insbesondere verwirklicht werden durch:
    \begin{itemize}
      \item die Bereitstellung und Pflege einer Räumlichkeit sowie der zur
        Verwirklichung der Vereinszwecke nötigen Infrastruktur.

      \item die Organisation und Durchführung von lokalen Zusammenkünften und
        Informationsveranstaltungen sowie Öffentlichkeitsarbeit.

      \item die Zusammenarbeit und der Austausch mit nationalen und
        internationalen Gruppierungen, deren Ziele mit denen des Vereins
        vereinbar sind.
    \end{itemize}
\end{enumerate}


\section{Selbstlosigkeit und Gemeinnützigkeit}
\begin{enumerate}
  \item Der Verein ist selbstlos tätig; er verfolgt ausschließlich und
    unmittelbar gemeinnützige Zwecke im Sinne des Abschnitts
    \glqq steuerbegünstigte Zwecke \grqq der Abgabenordnung und ist nicht auf
    eigenwirtschaftliche Zwecke ausgerichtet.

  \item Mittel der Körperschaft dürfen nur für die satzungsmäßigen Zwecke
    verwendet werden.

  \item Die Mitglieder erhalten in ihrer Eigenschaft als Mitglieder keine
    Zuwendungen aus Mitteln der Körperschaft.

  \item Es darf keine Person durch Ausgaben, die dem Zweck der Körperschaft
    fremd sind, oder durch unverhältnismäßig hohe Vergütungen begünstigt werden.
\end{enumerate}


\section{Mitgliedschaft}
\begin{enumerate}
  \item Mitglied des Vereins kann jede natürliche und juristische Person sein.

  \item Über den Aufnahmeantrag entscheidet der Vorstand. Gegen die Ablehnung
    steht der sich bewerbenden Person die Berufung an die Mitgliederversammlung
    zu, die schriftlich binnen eines Monats an den Vorstand zu richten ist.

  \item Der Aufnahmeantrag eines Minderjährigen bedarf der Zustimmung durch den
    gesetzlichen Vertreter. Mit Vollendung des 16. Lebensjahres haben
    jugendendliche Mitglieder ein Stimmrecht in der Mitgliederversammlung,
    soweit nicht der gesetzliche Vertreter des Minderjährigen - die mit dem
    Aufnahmeantrag als erteilt geltende - Einwilligung hierzu ausdrücklich
    widerrufen hat.

  \item Die Mitgliedschaft endet durch Austritt, Tod oder Ausschluss.

  \item Die schriftliche Austrittserklärung muss mit einer Frist von einem Monat
    jeweils zum Quartalsende gegenüber dem Vorstand erklärt werden.

  \item Ein Ausschluss kann nur aus wichtigem Grund erfolgen. Wichtige Gründe
    sind insbesondere, jedoch nicht abschließend:
    \begin{itemize}
      \item ein die Vereinsziele schädigendes Verhalten,

      \item die Verletzung satzungsmäßiger Pflichten,

      \item Beitragsrückstände von mindestens einem halben Jahr,

      \item strafrechtlich relevantes Verhalten.
    \end{itemize}

  \item Über den Ausschluss entscheidet der Vorstand. Der Ausschluss erfolgt
    unter Berücksichtigung einer Stellungnahme des Mitglieds, zu der dieses eine
    vierwöchige Frist erhält. Gegen den Ausschluss steht dem Mitglied die
    Berufung an die Mitgliederversammlung offen, die schriftlich binnen eines
    Monats an den Vorstand zu richten ist. Bis zu einer Entscheidung ruht die
    Mitgliedschaft.

  \item Mitglieder haben auf der Mitgliederversammlung Rede- und Antragsrecht,
    Stimmrecht sowie aktives und passives Wahlrecht.

  \item Die Mitgliederversammlung kann solche Personen, die sich besondere
    Verdienste um den Verein oder um die von ihm verfolgten satzungsgemäßen
    Zwecke erworben haben, zu Ehrenmitgliedern ernennen. Näheres kann in einer
    Ehrenordnung geregelt werden. Ehrenmitglieder haben alle Rechte eines
    ordentlichen Mitglieds.
\end{enumerate}


\section{Fördermitglieder}
\begin{enumerate}
  \item Fördermitglied des Vereins kann jede natürliche und juristische Person sein.

  \item Über den Aufnahmeantrag entscheidet der Vorstand. Gegen die Ablehnung
    steht der sich bewerbenden Person die Berufung an die Mitgliederversammlung
    zu, die schriftlich binnen eines Monats an den Vorstand zu richten ist.

  \item Der Aufnahmeantrag eines Minderjährigen bedarf der Zustimmung durch den
    gesetzlichen Vertreter.

  \item Die Fördermitgliedschaft endet durch Austritt, Tod oder Ausschluss.

  \item Die schriftliche Austrittserklärung muss mit einer Frist von einem Monat
    jeweils zum Quartalsende gegenüber dem Vorstand erklärt werden.

  \item Ein Ausschluss kann nur aus wichtigem Grund erfolgen. Wichtige Gründe
    sind insbesondere, jedoch nicht abschließend:
    \begin{itemize}
      \item ein die Vereinsziele schädigendes Verhalten,

      \item die Verletzung satzungsmäßiger Pflichten,

      \item Beitragsrückstände von mindestens einem halben Jahr,

      \item strafrechtlich relevantes Verhalten.
    \end{itemize}

  \item Über den Ausschluss entscheidet der Vorstand. Der Ausschluss erfolgt
    unter Berücksichtigung einer Stellungnahme des Fördermitglieds, zu der dieses eine
    vierwöchige Frist erhält. Gegen den Ausschluss steht dem Fördermitglied die
    Berufung an die Mitgliederversammlung offen, die schriftlich binnen eines
    Monats an den Vorstand zu richten ist. Bis zu einer Entscheidung ruht die
    Fördermitgliedschaft.

  \item Fördermitglieder haben auf der Mitgliederversammlung Rederecht und 
    Antragsrecht aber kein Stimmrecht, kein aktives Wahlrecht und kein passives Wahlrecht.
\end{enumerate}


\section{Beiträge}
\begin{enumerate}
  \item Der Verein kann Beiträge erheben.

  \item Höhe und Fälligkeit der Beiträge werden von der Mitgliederversammlung in
    der Beitragsordnung festgelegt.

  \item Der Vorstand kann in begründeten Einzelfällen bestimmen, dass der
    Beitrag in anderer Form als durch Geldzahlung erbracht wird oder
    Beitragsleistungen stunden.
\end{enumerate}


\section{Vorstand}
\begin{enumerate}
  \item Der Gesamtvorstand des Vereins besteht mindestens aus dem
    1. Vorsitzenden, dem 2. Vorsitzenden und dem Schatzmeister.

  \item Der vertretungsberechtigte Vorstand im Sinne des § 26 BGB besteht
    aus dem 1.  Vorsitzenden, 2. Vorsitzenden und dem Schatzmeister. Jeder von
    ihnen vertritt den Verein einzeln.

  \item Der Vorstand wird von der Mitgliederversammlung auf die Dauer von zwei
    Jahren gewählt; jedes Vorstandsmitglied bleibt jedoch so lange im Amt bis
    eine Neuwahl erfolgt ist.

  \item Es muss ein Protokoll bei der Vorstandssitzung geführt werden. Das
    Protokoll muss den Mitgliedern zugänglich gemacht werden.
\end{enumerate}


\section{Mitgliederversammlung}
\begin{enumerate}
  \item Die ordentliche Mitgliederversammlung findet einmal jährlich statt.
    Außerdem muss eine Mitgliederversammlung einberufen werden, wenn das
    Interesse des Vereins es erfordert oder wenn mindestens $\frac{1}{5}$ der
    Mitglieder die Einberufung schriftlich unter Angabe des Zwecks und der
    Gründe verlangt.

  \item Jede Mitgliederversammlung ist vom Vorstand schriftlich oder in Textform
    per E-Mail unter Einhaltung einer Einladungsfrist von 2 Wochen und unter
    Angabe der Tagesordnung einzuberufen.

  \item Zu Beginn jeder Mitgliederversammlung wird ein Schriftführer gewählt.

  \item Versammlungsleiter ist der 1. Vorsitzende und im Falle seiner
    Verhinderung der 2. Vorsitzende. Sollten beide nicht anwesend sein, wird ein
    Versammlungsleiter von der Mitgliederversammlung gewählt.

  \item Jede ordnungsgemäß einberufene Mitgliederversammlung ist ohne Rücksicht
    auf die Zahl der erschienenen Mitglieder beschlussfähig.

  \item Die Beschlüsse der Mitgliederversammlung werden mit einfacher Mehrheit
    der abgegebenen gültigen Stimmen gefasst. Zur Änderung der Satzung und des
    Vereinszwecks ist jedoch eine Mehrheit von $\frac{3}{4}$ der abgegebenen
    gültigen Stimmen erforderlich.

  \item Über die Beschlüsse der Mitgliederversammlung ist ein Protokoll
    aufzunehmen, das vom Versammlungsleiter und dem Schriftführer zu
    unterschreiben ist.
\end{enumerate}


\section{Auflösung, Anfall des Vereinsvermögens}
\begin{enumerate}
  \item Zur Auflösung des Vereins ist eine Mehrheit von $\frac{4}{5}$ der
    abgegebenen gültigen Stimmen erforderlich.

  \item Bei Auflösung oder Aufhebung des Vereins oder bei Wegfall
    steuerbegünstigter Zwecke fällt das Vermögen des Vereins an:
    Radio Unerhört Marburg e.V., Sitz in Marburg.
\end{enumerate}

\end{document}
